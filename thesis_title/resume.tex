%==========================================================================
%Template File for Monthly Lectual Meeting
%2006/05/22 (kkobayashi@mikilab.doshisha.ac.jp)
%==========================================================================

\documentclass[a4j,9pt,twocolumn]{jsarticle}
\usepackage{mlm2.0}
\usepackage{epsf}
\pagestyle{empty}


\begin{document}
\twocolumn[
%---------------------------------------------------------------------------
% ヘッダ    書式:\beginheader{回}{年}{月}
%---------------------------------------------------------------------------
\beginheader{2016}{12}{13}


%---------------------------------------------------------------------------
% 発表題目    書式:\title{日本語}{英語} 「\\」で改行できます
%---------------------------------------------------------------------------
\title
{モバイル端末内蔵の照度センサを用いた複数端末間の連携を実現する\\
\hspace{22mm} ハンドジェスチャ操作手法の検討}%

%---------------------------------------------------------------------------
% 著者名      書式:\author{日本語著者名}{英語著者名}
%---------------------------------------------------------------------------
\author{松井健人}
\studentid{3G150131}
\fellow{山下大輔}
%---------------------------------------------------------------------------


\endheader

\begin{abstract}
%---------------------------------------------------------------------------
近年,スマートフォンやタブレット端末の普及,スマートウォッチの登場により個人が所有するモバイル端末の台数が増加している.
そのため,モバイル端末間でのデータ共有やグルーピングなどに対して需要が高まっている.
しかし,これらの複数端末間の連携を実現するためには,複雑な操作が必要となる.
そこで我々は,複数端末間のインタラクションを可能とするためのハンドジェスチャ操作を実現する新しい技術として,モバイル端末内蔵の照度センサを用いた複数端末間の連携を実現するハンドジェスチャ操作手法(IllumiConnect)を提案する.
IllumiConnectは,モバイル端末の上で行われたジェスチャを認識するために,モバイル端末に内蔵された照度センサを用いる.
取得した照度値の履歴を基にした決定木学習を用いてハンドジェスチャを認識することで,直感的な操作を実現する.
IllumiConnectを評価するために,ハンドジェスチャ認識精度を検証するための評価実験およびデバイスのグルーピング精度を検証するための評価実験の2種類の評価実験を行った.
それぞれの実験によりでは,IllumiConnectは高い精度でそれぞれのジェスチャを認識,グルーピング可能であることがわかった.
%---------------------------------------------------------------------------
\end{abstract}
\vspace{3mm}

%---------------------------------------------------------------------------
% 本文
%---------------------------------------------------------------------------

{\large\bfseries 内容(章立て)}
\section{序論}
\section{複数端末間の連携を実現する直感的な操作手法}
\section{異機種におけるハンドジェスチャ認識に関する予備実験}
\subsection{予備実験概要}
\subsection{異機種における内蔵照度センサの性能検証実験}
\subsection{異機種におけるハンドジェスチャ認識精度の検証実験}
\section{モバイル端末内蔵の照度センサを用いた複数端末間の連携を実現するハンドジェスチャ操作手法の検討}
\subsection{提案手法の概要}
\subsection{提案手法で識別可能なジェスチャ種類}
\subsection{提案手法の構成}
\subsection{提案手法のアルゴリズム}
\section{モバイル端末内蔵の照度センサを用いた複数端末間の連携を実現するハンドジェスチャ操作手法の検討の評価実験}
\subsection{評価実験概要}
\subsection{評価実験環境}
\subsection{ハンドジェスチャ認識精度の検証実験}
\subsection{グルーピング精度の検証実験}
\section{結論}
]

%------------------------------------------------------------------------------
\end{document}
