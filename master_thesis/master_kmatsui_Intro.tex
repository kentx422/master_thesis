% !TEX root = master_kmatsui.tex

\chapter{序論}
近年,スマートフォンやタブレット端末の普及,またスマートウォッチの登場により個人が所有するモバイル端末の台数が増加している.
そのため,複数端末間の連携に対して需要が高まっている.
複数端末間の連携の例として,写真の送信が考えられる.
スマートフォンで撮影した写真を,友人と共有するために友人のスマートウォッチに送信する,あるいは大きな画面で見るために自身が所有するタブレット端末に送信するといった使い方である.
しかし,端末間での連携を行うためには,端末間でネットワークを構築する必要がある.
例えば,Bluetooth を利用する場合,使用する端末の周辺にある多くの端末が表示されたリストの中から,接続したい端末を探し出して選択する,接続する,接続を解除するなどの操作が必要である.
このような操作は,ユーザにとって煩雑であり,複数端末間の連携を躊躇する要因であると考えられる.
そのため,より直感的な操作手法が必要である.

直感的な操作手法として,NUI(Natural User Interface)が考えられる.
NUIとは,人間にとって自然かつ直感的な動作で操作が可能な入力手法であり,タッチ操作や音声操作などが一般的である.
NUIの中でも,特にジェスチャ操作に対して関心が高まっている.
ジェスチャ操作は,手や指,体の動きを認識して操作する手法で,主に非接触で操作できる.
ジェスチャ操作に関しては,様々実現方法が研究されている[].
また,ジェスチャ操作を実現するデバイスとして,Kinect[]やLeap Motion[]などが開発,商用化され,ジェスチャ操作はエンターテイメント分野や医療分野など広い分野において積極的に活用されている.
しかしジェスチャ操作は,一般的にジェスチャを認識するための特殊なデバイスが必要である.
モバイル端末におけるジェスチャ操作の実現を想定する際には,モバイル端末のメリットである携帯性を損なわないために,特殊なデバイスが不要であることが望ましい.

特殊なデバイスを用いずにモバイル端末におけるジェスチャ操作を実現する手法として,イメージセンサや加速度センサ,マイクロフォンなどモバイル端末内蔵のセンサを用いた手法が研究されている[].
これらの研究は,ジェスチャ操作の中でも,手の動きを認識して操作する,ハンドジェスチャ操作が多い.
例えば,イメージセンサを用いた手法[],加速度センサを用いた手法[],マイクロフォンを用いた手法[] などがある.
その他にも,特殊なペンデバイスを用いた手法[],特定の端末に内蔵されているホバー機能を用いた手法,同時に画面をタップする手法[] などがある.
これらの中には,1台の端末のみで完結する手法と複数端末

モバイル端末

これらの手法に対して,本研究では,モバイル端末に内蔵されている照度センサを用いて直感的なペアリングを実現する.
先行研究により,照度センサを用いてジェスチャ認識が出来ることがわかっている[12].そこで,本研究では,複数のモバイル端末においてジェスチャ認識を行いより直感的なペアリング手法を検討するとともに新たなジェスチャについて検証する.


\begin{figure}[htbp]
  \begin{center}
   \includegraphics[width=80mm]{img/empty_white.png}
   \vspace{-1mm}
   \caption{User is sending a photo from one device to another grouped devices by hand gesture}
   \label{example_of_IllumiConnect}
  \end{center}
\end{figure}

\begin{comment}

\end{comment}
