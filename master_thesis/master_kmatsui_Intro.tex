% !TEX root = master_kmatsui.tex

\chapter{序論}
近年,スマートフォンやタブレット端末の普及,またスマートウォッチの登場により個人が所有するモバイル端末の台数が増加している.
そのため,複数端末間の連携に対して需要が高まっている.
複数端末間の連携の例として,写真の共有が考えられる.
例えば,スマートフォンで撮影した写真を,友人に見せるために友人のスマートフォンに送信する,あるいは大きな画面で見るために自身が所有するタブレット端末に送信するといった使い方である.
しかし,端末間での連携を行うためには,端末間でネットワークを構築する必要があり,端末同士をペアリングしなければならない.
例えば,Bluetooth を利用する場合,使用する端末の周辺にある多くの端末が表示された
リストの中から,ペアリングしたい端末を探し出して選択する必要がある.

そこで,このモバイル端末間でのペアリングをより直感的に行う手法が研究されている.
それらの研究の多くは,モバイル端末に内蔵されているセンサを用いている.
例えば,イメージセンサを用いた手法,加速度センサを用いた手法,[3,4],マイクロフォンを用いた手法[5{7] などがある.
その他にも,特殊なペンデバイスを用いた手法[8, 9],特定の端末に内蔵されているホバー機能を用いた手法,同時に画面をタップする手法[11] がある.

これらの手法に対して,本研究では,モバイル端末に内蔵されている照度センサを用いて直感的なペアリングを実現する.
先行研究により,照度センサを用いてジェスチャ認識が出来ることがわかっている[12].そこで,本研究では,複数のモバイル端末においてジェスチャ認識を行いより直感的なペアリング手法を検討するとともに新たなジェスチャについて検証する.


\begin{figure}[htbp]
  \begin{center}
   \includegraphics[width=80mm]{img/empty_white.png}
   \vspace{-1mm}
   \caption{User is sending a photo from one device to another grouped devices by hand gesture}
   \label{example_of_IllumiConnect}
  \end{center}
\end{figure}

\begin{comment}

\end{comment}
