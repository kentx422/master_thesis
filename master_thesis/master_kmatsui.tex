%----------------------------------------------------------------------
% 修士論文スタイルファイルのサンプル
% 修士論文:ICカードによる在席管理を導入した知的照明システムの省エネルギー性能
% 同志社大学大学院工学研究科情報工学専攻 知的システムデザイン研究室
% 博士前期課程2010年度0792番
% 米本 洋幸(hyonemoto@mikilab.doshisha.ac.jp)
% 2012/01/19 提出予定
%----------------------------------------------------------------------

%----------------------------------------------------------------------
%プリアンブルの指定です.
%11ptでjreportクラスを使用しています.
%ですから,一番大きな見出しがchapterになります.
%注意してください.
%----------------------------------------------------------------------
\documentclass[a4paper,11pt]{jreport}
\usepackage{masterthesis}
\usepackage{latexsym}
\usepackage{amsmath,amssymb}
\usepackage{multirow}
\usepackage{comment}
%\usepackage{showkeys}%推敲用
\pagestyle{plain}
\bibliographystyle{junsrt}
\makeatletter
\def\@cite#1{$\m@th^{\hbox{\@ove@rcfont #1)}}$}
\def\Underline{\setbox0\hbox\bgroup\let\\\endUnderline}
\def\endUnderline{\vphantom{y}\egroup\smash{\underline{\box0}}\\}
\def\|{\verb|}
\def\newblock{\hskip .11em plus .33em minus .07em}

\newcommand\figref[1]{\fgref{#1}}
\newcommand\tabref[1]{\tbref{#1}}
%デバッグ用
%\newcommand\memo[1]{{\Huge #1}}
\newcommand\memo[1]{}

\makeatother
\begin{document}

%----------------------------------------------------------------------
% 表紙
%----------------------------------------------------------------------

\begin{toppage}

% 修士論文のタイトル
% 改行を入れる場合には \\ を入力
% 第2引数はレイアウト調整のためのパラメータ.タイトルと氏名までの間隔を
% 表している.推奨値は以下の通り.
%   タイトルが1行の時:9
%   タイトルが2行の時:8
%   タイトルが3行の時:7

\title{モバイル端末内蔵の照度センサを用いた\\複数端末間の連携を実現する\\ハンドジェスチャ操作手法の検討}{Hand Gesture Operation using built-in Illuminance meter\\to Enable Multi-Device Interaction}{4}

% 学籍番号・著者氏名
% 名字と名前の間に全角スペースをいれること
% \author{ 入学年 }{ 番号(専攻内) }{ 氏名 }

\author{2015}{131}{松井 健人}

% 指導教官
\professor{三木 光範教授}

% 提出日
\date{2017}{1}{21}

\end{toppage}

%----------------------------------------------------------------------
% abstract
%----------------------------------------------------------------------
\begin{abstract}
% !TEX root = master_kmatsui.tex
In this paper, we present IllumiConnect, a novel technique for realize a hand gesture operation to enable multi-device interaction.
IllumiConnect uses illuminance meter built into mobile devices such as smartphones and tablet terminal to sense hand gestures performed over between mobile devices.
By recognizing hand gestures using decision tree learning based on  history of illuminance value, we realize a intuitive operation by hand gestures over mobile devices.
We conducted two kind of experiments for evaluating IllumiConnect.
In an evaluation experiment for accuracy of hand gestures recognition, IllumiConnect showed to be able to distinguish each one of hand gestures high accuracy.
And in an evaluation experiment for verifying accuracy of grouping some devices, IllumiConnect showed to be able to group devices almost correct.
We also implemented a application and introduce a specific use case of IllumiConnect.

\begin{comment}
  日本語
  本稿では,私たちはIllumiConnect,複数端末間のインタラクションを可能とするためのハンドジェスチャ操作を実現する新しい技術,を示す.
  IllumiConnectは,モバイル端末の上で行われたジェスチャを認識するために,スマートフォンやタブレット端末などのモバイル端末に内蔵された照度センサを用いる.
  照度値の履歴を基にした決定木学習を用いてハンドジェスチャを認識することで,私たちはモバイル端末上でのハンドジェスチャによる直感的な操作を実現す.
  私たちは,IllumiConnectを評価するために二種類の評価実験を行った.
  ハンドジェスチャ認識精度を検証するための評価実験では,IllumiConnectは高い精度でそれぞれのジェスチャを認識可能であることを示した。
  また,任意デバイスのグルーピング化の精度を検証するための評価実験では,IllumiConnectは,ほとんど正確に端末をグルーピングできることを示した.
  また,私たちは,アプリケーションを作成し, IllumiConnectの具体的な使用例を紹介する.
\end{comment}

%-----------------------
%近年,ほとんどの人がスマートフォンを所持している.
%また,1人の人がスマートフォンやタブレット,スマートウォッチなど複数のモバイル端末を所有している.
%このようにモバイル端末が多くなってきているので,モバイル端末間でのデータ共有やグルーピングなどに対しての需要が高まっている.
%しかし,それらを行うためには,複雑な操作が必要となる.
%例えば,近くにあるデバイスを検出し,送りたいデバイスを選択する.
%そして,そのデバイスと接続できたら,送りたいデータを選択しそのデバイスに送信する.
%そこで,この問題を解消するために直感的な操作によりデータ送信やグルーピングを行う研究がたくさん行われている.
%しかし,これらの操作にも問題はある.
%そこで,本稿では,モバイル端末に内蔵されている照度センサを用いて,複数端末の連携をハンドジェスチャによる直感的な操作により実現するIllumiConnectを作成した.
%IllumiConnectのジェスチャ操作の認識精度の検証と複数のモバイル端末でのグルーピング精度の検証を行うために被験者実験を行った.
%その結果,IllumiConnectはジェスチャ操作を高い精度で認識し、また、ほとんど確実にグルーピングできることがわかった.

\end{abstract}

%----------------------------------------------------------------------
% 目次
%----------------------------------------------------------------------
\contents
%
%----------------------------------------------------------------------
% 本文
%----------------------------------------------------------------------

% 序論
% !TEX root = master_kmatsui.tex

\chapter{序論}
近年,スマートフォンやタブレット端末の普及,またスマートウォッチの登場により個人が所有するモバイル端末の台数が増加している.
そのため,複数端末間の連携に対して需要が高まっている.
複数端末間の連携の例として,写真の送信が考えられる.
スマートフォンで撮影した写真を,友人と共有するために友人のスマートウォッチに送信する,あるいは大きな画面で見るために自身が所有するタブレット端末に送信するといった使い方である.
しかし,端末間での連携を行うためには,端末間でネットワークを構築する必要がある.
例えば,Bluetooth を利用する場合,使用する端末の周辺にある多くの端末が表示されたリストの中から,接続したい端末を探し出して選択する,接続する,接続を解除するなどの操作が必要である.
このような操作は,ユーザにとって煩雑であり,複数端末間の連携を躊躇する要因であると考えられる.
そのため,より直感的な操作手法が必要である.

直感的な操作手法として,NUI(Natural User Interface)が考えられる.
NUIとは,人間にとって自然かつ直感的な動作で操作が可能な入力手法であり,タッチ操作や音声操作などが一般的である.
NUIの中でも,特にジェスチャ操作に対して関心が高まっている.
ジェスチャ操作は,手や指,体の動きを認識して操作する手法で,主に非接触で操作できる.
ジェスチャ操作に関しては,様々実現方法が研究されている[].
また,ジェスチャ操作を実現するデバイスとして,Kinect[]やLeap Motion[]などが開発,商用化され,ジェスチャ操作はエンターテイメント分野や医療分野など広い分野において積極的に活用されている.
しかしジェスチャ操作は,一般的にジェスチャを認識するための特殊なデバイスが必要である.
モバイル端末におけるジェスチャ操作の実現を想定する際には,モバイル端末のメリットである携帯性を損なわないために,特殊なデバイスが不要であることが望ましい.

特殊なデバイスを用いずにモバイル端末におけるジェスチャ操作を実現する手法として,イメージセンサや加速度センサ,マイクロフォンなどモバイル端末内蔵のセンサを用いた手法が研究されている[].
これらの研究は,ジェスチャ操作の中でも,手の動きを認識して操作する,ハンドジェスチャ操作が多い.
例えば,イメージセンサを用いた手法[],加速度センサを用いた手法[],マイクロフォンを用いた手法[] などがある.
その他にも,特殊なペンデバイスを用いた手法[],特定の端末に内蔵されているホバー機能を用いた手法,同時に画面をタップする手法[] などがある.
これらの中には,1台の端末のみで完結する手法と複数端末

モバイル端末

これらの手法に対して,本研究では,モバイル端末に内蔵されている照度センサを用いて直感的なペアリングを実現する.
先行研究により,照度センサを用いてジェスチャ認識が出来ることがわかっている[12].そこで,本研究では,複数のモバイル端末においてジェスチャ認識を行いより直感的なペアリング手法を検討するとともに新たなジェスチャについて検証する.


\begin{figure}[htbp]
  \begin{center}
   \includegraphics[width=80mm]{img/empty_white.png}
   \vspace{-1mm}
   \caption{User is sending a photo from one device to another grouped devices by hand gesture}
   \label{example_of_IllumiConnect}
  \end{center}
\end{figure}

\begin{comment}

\end{comment}

\newpage
% 関連研究
% !TEX root = master_kmatsui.tex
\chapter{複数端末間の連携を実現する直感的な操作手法}
関連研究

\newpage
% 予備実験
% !TEX root = master_kmatsui.tex
\chapter{異機種におけるハンドジェスチャ認識に関する予備実験}

\section{異機種におけるハンドジェスチャ認識に関する予備実験の概要}

\section{異機種における内蔵照度センサの性能検証実験}
\subsection*{異機種における内蔵照度センサの性能検証実験の概要}
\subsection*{異機種における内蔵照度センサの性能検証実験の結果}
\subsection*{異機種における内蔵照度センサの性能検証実験の考察}
\section{異機種におけるハンドジェスチャ認識精度の検証実験}
\subsection*{異機種におけるハンドジェスチャ認識精度の検証実験の概要}
\subsection*{異機種におけるハンドジェスチャ認識精度の検証実験の結果}
\subsection*{異機種におけるハンドジェスチャ認識精度の検証実験の考察}

\newpage
% 提案システム
% !TEX root = master_kmatsui.tex
\chapter{IllumiConnect}
\section{IllumiConnectの概要}
\section{識別可能なジェスチャの種類}
\section{IllumiConnectの構成図}
\section{IllumiConnectのアルゴリズム}

\newpage
% 本実験
% !TEX root = master_kmatsui.tex
\chapter{IllumiConnectの評価実験}
\section{IllumiConnectの評価実験の概要}
\section{IllumiConnectの評価実験の環境}
\section{IllumiConnectにおけるハンドジェスチャ認識精度の検証実験}
\subsection*{IllumiConnectにおけるハンドジェスチャ認識精度の検証実験の概要}
\subsection*{IllumiConnectにおけるハンドジェスチャ認識精度の検証実験の結果}
\subsection*{IllumiConnectにおけるハンドジェスチャ認識精度の検証実験の考察}
\section{IllumiConnectにおけるグルーピング精度の検証実験}
\subsection*{IllumiConnectにおけるグルーピング精度の検証実験の概要}
\subsection*{IllumiConnectにおけるグルーピング精度の検証実験の結果}
\subsection*{IllumiConnectにおけるグルーピング精度の検証実験の考察}

\newpage
% 使用例
% !TEX root = master_kmatsui.tex
\chapter{IllumiConnectのユースケース}
\section{IllumiConnectのユースケースの概要}
\section{IllumiConnectを用いたデータ共有アプリケーション}

\newpage
% 結論
% !TEX root = master_kmatsui.tex
\chapter{結論}

\newpage
% 今後の展望
% !TEX root = master_kmatsui.tex
\chapter{今後の展望}

\newpage

\thispagestyle{empty}
%----------------------------------------------------------------------
%謝辞
%----------------------------------------------------------------------
\begin{gratitude}
本研究を遂行するにあたり,多大なる御指導そして御協力を頂きました,同志社大学理工学部の三木光範教授に心より感謝致します.また,貴重な御指摘,御助言を頂きました,同志社大学理工学部の間博人助教に心より感謝致します.

本研究を実施および本論文の執筆にあたり,多大な御協力を頂きました,知的システムデザイン研究室の山下大輔氏,村上広記氏,相馬啓佑氏に心より感謝致します.

快適な研究生活を送るために様々な御助力を頂きました,正木羊子氏に心より感謝致します.

\end{gratitude}

\newpage
\thispagestyle{empty}
%----------------------------------------------------------------------
%参考文献の記入欄です.
%----------------------------------------------------------------------

\bibliography{bibsample}

\thispagestyle{empty}
%----------------------------------------------------------------------
%以下,図表のためのページです.
%再びページ番号を振り直しています.
%論文が完成すると,本文と図表の間に色紙を挟みます.
%----------------------------------------------------------------------

% 付図・付表のリストを作成する
%\appendixpage{1}

%----------------------------------------------------------------------
%図の張り込み例です.
%----------------------------------------------------------------------

%\input{master_Appendix}

\end{document}
